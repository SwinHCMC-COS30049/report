
\section{Background}
Over the past decade, "Blockchain" has become a ubiquitous term across media, popular discourse, and academic literature, particularly following Satoshi Nakamoto's introduction of Bitcoin cryptocurrency in 2008 \cite{wright2008bitcoin}. Despite its widespread recognition, blockchain remains a relatively nascent technology whose practical applications continue to evolve and are often misunderstood. A common misconception equates blockchain with Bitcoin, when in fact, blockchain technology serves as the fundamental infrastructure that enables Bitcoin and numerous other applications.

At its core, blockchain functions as a distributed digital record-keeping system that ensures transactions cannot be secretly altered or tampered with. Unlike traditional systems that rely on central authorities like banks or governments, blockchains operate through a network of users who collectively maintain and verify the shared ledger. The technology works by grouping cryptographically signed transactions into blocks, with each new block linking to its predecessor through sophisticated cryptographic methods. This creates an increasingly secure chain where older records become progressively harder to alter, while multiple copies of the ledger exist throughout the network with built-in protocols that automatically resolve any discrepancies \cite{yaga2018blockchain}.

\section{Context}
Blockchain technology, as implemented in Bitcoin, facilitates peer-to-peer transactions characterized by decentralization, cryptographic security, and transparency. This distributed ledger system enables various financial operations, including e-commerce transactions, digital asset custody, and value transfer, without traditional intermediary institutions. The technology presents significant advantages in the business domain, particularly in terms of operational transparency, cryptographic security protocols, enhanced auditability, and optimized transaction efficiency with reduced costs. While these inherent benefits have attracted substantial global investment in Bitcoin and alternative cryptocurrencies, the technological complexity presents a significant barrier to comprehensive user understanding. Despite the inherent transparency of blockchain networks and their public transaction records, the technical sophistication of the data architecture poses considerable challenges for interpretation. Consequently, there exists a critical imperative for the development of sophisticated visualization tools that can facilitate intuitive analysis of transaction patterns, network metrics, and smart contract interactions, thereby democratizing access to blockchain data analytics.