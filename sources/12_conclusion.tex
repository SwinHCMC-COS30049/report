\section{Discussion}
There were some insightful observations along the development process. The directed graph approach was highly suitable for representing blockchain transaction data, as it inherently illustrates the flow of assets between addresses. This was found more intuitive by users than conventional tabular representations. Though graph databases facilitated efficient retrieval of data for visualization, optimization was inevitable to handle large numbers of transactions. Pagination and selective querying were thus introduced to preserve responsiveness.

Achieving an effective balance between interactivity and simplicity was a main challenge. Supporting navigation between transactions with click-to-expand functionality helped to add depth without overwhelming users with too much information. Visualizing tabular transaction information in the visual graph was essential since it allowed users to see significance in relationships and thus showed how visual and textual modes work best in collaboration. Keeping data current with blockchain data through blockchain node synchronisation was crucial and highlighted the role of real-time data in visualisations for finance.

The system proves the capability of visualization tools to render blockchain technology more understandable to non-technical individuals, thereby potentially enhancing the adoption and comprehension of distributed ledger technologies.

\section{Conclusion}
The Blockchain Transaction Data Visualization System successfully translates complex blockchain transaction data into a usable and understandable form. By using directed graph visualizations in addition to detailed tabular data, the system helps users infer conclusions about transactional patterns and relationships that could otherwise be hidden in raw blockchain data.

The graph database back end has proved to be effective as a technological solution to provide required performance and relationship-based data model to enable effective visualization for transaction networks. Features to enable interactive exploration and navigation between transaction hops have efficiently highlighted interconnectedness within blockchain ecosystems.

Future work could improve the system's functionality with the addition of machine learning algorithms to detect suspicious patterns in transactions. Additionally, adding features for temporal analysis would allow for visualizing transaction flows through time. Cross-chain visualization support would provide for linking transactions between different blockchains. Lastly, using sophisticated filtering techniques could provide for a focused examination on certain types or value ranges for transactions.

This project demonstrates the immense power that good visualization can have in enhancing usability and accessibility for blockchain information, and thus encourage wider adoption and understanding for distributed ledger technologies. With applications for blockchain going far beyond digital currency, such visualization software is going to be ever more useful to users in a wide range of industries, from finance to supply chain to digital asset management.